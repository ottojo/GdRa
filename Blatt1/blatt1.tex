\documentclass[a4paper]{article}

\usepackage[ngerman]{babel}
\usepackage[utf8]{inputenc}
\usepackage{enumitem}
\usepackage{dirtytalk}

\author{Jonas Otto}
\title{Übungsblatt 1}

\begin{document}

\maketitle

\section{Einleitungsfragen}
\begin{enumerate}[label=\alph*)]
    \item Unter Rechnerarchitektur stelle ich mir das Aufbauen von Prozessoren und
        Prozessorsystemen aus immer kleineren Baugruppen bis zu Logikgattern hin vor.

    \item Unter den Grundlagen der Rechnerarchitektur stelle ich mir Grundlagen der boolschen
        Algebra, Theorie zum Aufbau von Recheneinheiten und Theorie zur Programmierung und
        Programmausführung von Prozessoren (Assemblerprogrammierung) vor. Auch bestehende 
        Architekturen von Prozessoren gehören für mich zu den Grundlagen der Rechnerarchitektur.

    \item \say{Was würdet ihr machen, wenn ich hier einfach hinschreiben würde: \say{Ich gehe ohne
            Erwartungen oder Hoffnungen in die Vorlesung} ? Wäre das dann falsch? Nur so als 
        Gedanke am Rande...}\footnote{Tim Luchterhand, GdRa Übungsblatt 0, 2018}\\
        Ich erwarte von der Vorlesung, nach diesem Semester ein tieferes Verständnis vom
        Aufbau moderner Prozessoren und Rechner zu haben. 

\end{enumerate}


\section{Historische Entwickling}
\begin{enumerate}[label=\alph*)]
    \item Mechanische Rechner haben bewegliche Teile, die schnell zu Verschleiß führen. Außerdem
        sind diese dadurch stark in der möglichen Rechengeschwindigkeit begrenzt.
    \item Vor der elektronischen Speicherung mussten Daten machanisch gespeichert werden, z.B.
        mittels Lochkarten. Diese wurden erstmals in den 1890er Jahren eingesetzt, zum Zwecke der
        Datenspeicherung bei einer Volkszählung.
    \item Die grundlegende Architektur früher Rechner unterschied sich stark, sodass bestehende
        Programme mit dem Befehlssatz des neuen Rechners erneut implementiert werden mussten.
    \item Elektromechanische Rechner wurden zuerst durch Rechner ersetzt, die Vakuumröhren
        als Schaltelemente nutzen. Seit der Erfindung des Transistors werden Rechner vollständig
        mittels Halbleitertechnologie realisiert.
    \item Die im kaufmännische Bereich eingesetzten Rechenmaschienen wurden primär für die
        Buchführung ausgelegt und konstruiert, was deutlich weniger Rechenleistung erforderte
        als wissenschaftliche Berechnungen, was sie für diese untauglich machte.

\end{enumerate}


\end{document}
