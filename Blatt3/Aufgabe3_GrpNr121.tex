\documentclass[a4paper]{article}

\usepackage[ngerman]{babel}
\usepackage[utf8]{inputenc}
\usepackage{enumitem}
\usepackage{amsmath}
\usepackage{graphicx}
\usepackage{mathabx}
\usepackage{todonotes}


\title{Grundlagen der Rechnerarchitektur\\ Übungsblatt 3\\Gruppe 121\\}
\author{ Jonas Otto\\ \and Dominik Authaler \\ 
}

\date{\today}

\begin{document}

\maketitle

\section{Dezimalzahlen umrechnen}

\begin{enumerate}[label=\alph*)]
    \item
        \begin{equation*}
            1944_{10} = (1024 + 512 + 256 + 128 + 16 + 8)_{10} = 11110011000_{2}
        \end{equation*}
    \item
        \begin{equation*}
            1476_{10} = (2 \cdot 512 + 7 \cdot 64 + 4 \cdot 1)_{10} = (2 \cdot 8^3 + 7 \cdot 8^2 + 4 \cdot 8^0)_{10} = 2704_{8}
        \end{equation*}
    \item  
        \begin{equation*}
        	1535_{10} = (5 \cdot 256 + 15 \cdot 16 + 15 \cdot 1)_{10} = (5 \cdot 16^2 + 15 \cdot 16^1 + 15 \cdot 16^0)_{10} = \text{5FF}_{16}
        \end{equation*}	
    \item 
        \begin{equation*}
        116_{10} = (2 \cdot 49 + 2 \cdot 7 + 4 \cdot 1)_{10} = (2 \cdot 7^2 + 2 \cdot 7^1 + 4 \cdot 7^0)_{10} = 224_{7} 
        \end{equation*}	
\end{enumerate}

\section{Ins Dezimalsystem umrechnen}
\begin{enumerate}[label=\alph*)]
    \item
        \begin{equation*}
        11000111_{2} = (1 + 2 + 4 + 64 + 128)_{10} = (199)_{10}
        \end{equation*}	
    \item
        \begin{equation*}
        1065_{7} = (1 \cdot 7^3 + 0 \cdot 7^2 + 6 \cdot 7^1 + 5 \cdot 7^0)_{10} = (343 + 42 + 5)_{10} = 390_{10}
        \end{equation*}	
    \item
        \begin{equation*}
        \text{2EA}_{16} = (2 \cdot 16^2 + 14 \cdot 16^1 + 10 \cdot 16^0)_{10} = (512 + 224 + 10)_{10} = 746_{10}
        \end{equation*}	
    \item
        \begin{equation*}
        3262_{8} = (3 \cdot 8^3 + 2 \cdot 8^2 + 6 \cdot 8^1 + 2 \cdot 8^0)_{10} = (1536 + 128 + 48 + 2)_{10} = 1714_{10}
        \end{equation*}	
\end{enumerate}

\section{Zwischen Systemen umrechnen}
\begin{enumerate}[label=\alph*)]
	\item 
		\begin{equation*}
		227_{16} = 001 000 100 111_{2} = 1047_{8}
		\end{equation*}
	\item 
		\begin{equation*}
		10 010 001 101_{2} = 2215_{8}
		\end{equation*}
	\item 
		\begin{equation*}
		1010 1111 1111 1110 1101 0000 0000 1111_{2} = AFFED00F_{16}
		\end{equation*}
	\item 
		Wegen $9 = 3^2$ wird jedes Symbol des 9er-Systems durch zwei Symbole des 3er-Systems ersetzt. 
		\begin{equation*}
		5742_{9} = 12 21 11 02_{3}
		\end{equation*}
\end{enumerate}

\section{Komisches Zahlensystem}
\begin{table}[h!]
	\begin{tabular}{|l|c|c|c|c|c|c|c|c|c|c|c|c|c|c|c|}
		\hline
			Dezimal-Wert & 0 & 1 & 2 & 3 & 4 & 5 & 6 & 7 & 8 & 9 & 10 & 11 & 12 & 13 & 14 \\		
		\hline
		Symbol & 0 & 1 & 2 & 3 & 4 & 5 & 6 & 7 & 8 & 9 & A & B & C & D & E \\
		\hline
	\end{tabular}

	\begin{tabular}{|l|c|c|c|c|c|c|}
		\hline
		Dezimal-Wert & 15 & 16 & 17 & 18 & 19 & 20 \\	
		\hline
		Symbol & F & G & H & I & J & K \\
		\hline
	\end{tabular}
\end{table}

\begin{enumerate}[label=\alph*)]
	\item
	\begin{equation*}
		26_{10} = (21 + 5)_{10} = (1 \cdot 21^1 + 5 \cdot 21^0)_{10} = 15_{21}
	\end{equation*}
	
	\item 
	\begin{equation*}
		19_{10} = (19 \cdot 21^0)_{10} = J_{21}
	\end{equation*}
\end{enumerate}

\section{Most Significant Bit}
\begin{enumerate}[label=\alph*)]
        \item MSB links:  $1050_8 = (1 \cdot 8^3 + 5 \cdot 8^1)_{10} = 552_{10}$ \\
		  MSB rechts: $1050_8 = (5 \cdot 8^2 + 1 \cdot 8^0)_{10} = 321_{10}$
		  
	\item MSB links: $10110010010_2 = (2^1 + 2^4 + 2^7 + 2^8 + 2^{10})_{10} = 1426_{10}$ \\
		MSB rechts: $10110010010_2 = (2^0 + 2^2 + 2^3 + 2^6 + 2^9)_{10} = 589_{10}$
	
	\item MSB links: $4242_{10} = 4242_{10}$ \\
		  MSB rechts: $4242_{10} = 2424_{10}$ 
		  
 	\item MSB links: $A47_{14} = (7 \cdot 14^0 + 4 \cdot 14^1 + 10 \cdot 14^2)_{10} = 2023_{10} = 133213_4$
 	\begin{align*}
 		2023 : 4 &= 505& R3 \\
 		505 : 4 &= 126& R1 \\
 		126 : 4 &= 31& R2 \\
 		31 : 4 &= 7& R3 \\
 		7 : 4 &= 1& R3 \\
 		1 : 4 &= 0& R1
 	\end{align*}
\end{enumerate}
		MSB rechts: $A47_{14} = (7 \cdot 14^2 + 4 \cdot 14^1 + 10 \cdot 14^0)_{10} = 1438_{10} = 112132_{4}$
		\begin{align*}
		1438 : 4 &= 359& R2 \\
		359 : 4 &= 89& R3 \\
		89 : 4 &= 22& R1 \\
		22 : 4 &= 5& R2 \\
		5 : 4 &= 1& R1 \\
		1 : 4 &= 0& R1
		\end{align*}
		
\section{Knobelaufgabe}
$65243_b \stackrel{!}{=} 27299_{10}$, wegen $65424_{10} > 27299_{10}$ muss $0 < b < 10$ gelten. Im Folgenden werden daher die wenigen möglichen Basen ausprobiert:
\begin{align*}
	65243_{9} = (6\cdot9^4 + 5 \cdot 9^3 + 2 \cdot 9^2 + 4 \cdot 9^1 + 3 \cdot 9^0)_{10} = 43212_{10}. \\
	65243_{8} = (6\cdot8^4 + 5 \cdot 8^3 + 2 \cdot 8^2 + 4 \cdot 8^1 + 3 \cdot 8^0)_{10} = 27299_{10}. 
\end{align*}
Die gesuchte Basis ist also $b = 8$.

\section{Festkomma}
	\begin{enumerate}[label=\alph*)]
		\item 
		$10,625_{10} = (8 + 2 + 0,5 + 0,125)_{10} = (2^3 + 2^1 + 2^{-1} + 2^{-3})_{10} = 1010,101_2$
		
		\item 
		$101101,1101_2 = (2^5 + 2^3  + 2^2 + 2^0 + 2^{-2} + 2^{-2} + 2^{-4}) = 45,8125_{10}$
	\end{enumerate}
\end{document}
	
