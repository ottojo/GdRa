\documentclass[a4paper]{article}

\usepackage[ngerman]{babel}
\usepackage[utf8]{inputenc}
\usepackage{enumitem}
\usepackage{amsmath}
\usepackage{graphicx}
\usepackage{mathabx}
\newcommand\mathbfont{\usefont{U}{mathb}{m}{n}}


\title{Grundlagen der Rechnerarchitektur\\ Übungsblatt 3\\Gruppe 121\\}
\author{ Jonas Otto\\ \and Dominik Authaler \\ 
}

\date{\today}


\def\mayaexpansion{%
	\mayacntc=\mayacnta\mathbfont
	\ifnum\mayacntc=0 0\else
	\rotatebox[origin=c]{-90}{%
		\loop\ifnum\mayacntc>5\advance\mayacntc by -5\repeat
		\the\mayacntc\mayacntc=\mayacnta
		\loop\ifnum\mayacntc>5\advance\mayacntc by -5 5\repeat}%
	\fi}%

\begin{document}

\maketitle

\section{Dezimalzahlen umrechnen}

\begin{enumerate}[label=\alph*)]
    \item
        \begin{equation*}
            1944_{10} = (1024 + 512 + 256 + 128 + 16 + 8)_{10} = 11110011000_{2}
        \end{equation*}
    \item
        \begin{equation*}
            1476_{10} = (2 \cdot 512 + 7 \cdot 64 + 4 \cdot 1)_{10} = (2 \cdot 8^3 + 7 \cdot 8^2 + 4 \cdot 8^0)_{10} = 2704_{8}
        \end{equation*}
    \item  
        \begin{equation*}
        	1535_{10} = (5 \cdot 256 + 15 \cdot 16 + 15 \cdot 1)_{10} = (5 \cdot 16^2 + 15 \cdot 16^1 + 15 \cdot 16^0)_{10} = \text{5FF}_{16}
        \end{equation*}	
    \item 
        \begin{equation*}
        116_{10} = (2 \cdot 49 + 2 \cdot 7 + 4 \cdot 1)_{10} = (2 \cdot 7^2 + 2 \cdot 7^1 + 4 \cdot 7^0)_{10} = 224_{7} 
        \end{equation*}	
\end{enumerate}

\section{Ins Dezimalsystem umrechnen}
\begin{enumerate}[label=\alph*)]
    \item
        \begin{equation*}
        11000111_{2} = (1 + 2 + 4 + 64 + 128)_{10} = (199)_{10}
        \end{equation*}	
    \item
        \begin{equation*}
        1065_{7} = (1 \cdot 7^3 + 0 \cdot 7^2 + 6 \cdot 7^1 + 5 \cdot 7^0)_{10} = (343 + 42 + 5)_{10} = 390_{10}
        \end{equation*}	
    \item
        \begin{equation*}
        \text{2EA}_{16} = (2 \cdot 16^2 + 14 \cdot 16^1 + 10 \cdot 16^0)_{10} = (512 + 224 + 10)_{10} = 746_{10}
        \end{equation*}	
    \item
        \begin{equation*}
        3262_{8} = (3 \cdot 8^3 + 2 \cdot 8^2 + 6 \cdot 8^1 + 2 \cdot 8^0)_{10} = (1536 + 128 + 48 + 2)_{10} = 1714_{10}
        \end{equation*}	
\end{enumerate}

\section{Zwischen Systemen umrechnen}
\begin{enumerate}[label=\alph*)]
	\item 
		\begin{equation*}
		227_{16} = 001 000 100 111_{2} = 1047_{8}
		\end{equation*}
	\item 
		\begin{equation*}
		10 010 001 101_{2} = 2215_{8}
		\end{equation*}
	\item 
		\begin{equation*}
		1010 1111 1111 1110 1101 0000 0000 1111_{2} = AFFED00F_{16}
		\end{equation*}
	\item 
		Wegen $9 = 3^2$ wird jedes Symbol des 9er-Systems durch zwei Symbole des 3er-Systems ersetzt. 
		\begin{equation*}
		5742_{9} = 12 21 11 02_{3}
		\end{equation*}
\end{enumerate}

\section{Komisches Zahlensystem}
\begin{table}[h!]
	\begin{tabular}{|l|c|c|c|c|c|c|c|c|c|c|c|c|c|c|c|}
		\hline
			Dezimal-Wert & 0 & 1 & 2 & 3 & 4 & 5 & 6 & 7 & 8 & 9 & 10 & 11 & 12 & 13 & 14 \\		
		\hline
		Symbol & 0 & 1 & 2 & 3 & 4 & 5 & 6 & 7 & 8 & 9 & A & B & C & D & E \\
		\hline
	\end{tabular}

	\begin{tabular}{|l|c|c|c|c|c|c|}
		\hline
		Dezimal-Wert & 15 & 16 & 17 & 18 & 19 & 20 \\	
		\hline
		Symbol & F & G & H & I & J & K \\
		\hline
	\end{tabular}
\end{table}

\begin{enumerate}[label=\alph*)]
	\item
	\begin{equation*}
		26_{10} = (21 + 5)_{10} = (1 \cdot 21^1 + 5 \cdot 21^0)_{10} = 15_{21}
	\end{equation*}
	
	\item 
	\begin{equation*}
		19_{10} = (19 \cdot 21^0)_{10} = J_{21}
	\end{equation*}
\end{enumerate}
\end{document}
	
