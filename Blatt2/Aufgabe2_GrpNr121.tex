\documentclass[a4paper]{article}

\usepackage[ngerman]{babel}
\usepackage[utf8]{inputenc}
\usepackage{enumitem}
\usepackage{amsmath}
\usepackage{graphicx}
\usepackage{mathabx}
\newcommand\mathbfont{\usefont{U}{mathb}{m}{n}}


\title{Grundlagen der Rechnerarchitektur\\ Übungsblatt 1\\Gruppe 121\\}
\author{ Jonas Otto\\ \and Dominik Authaler \\ 
}

\date{\today}


\def\mayaexpansion{%
	\mayacntc=\mayacnta\mathbfont
	\ifnum\mayacntc=0 0\else
	\rotatebox[origin=c]{-90}{%
		\loop\ifnum\mayacntc>5\advance\mayacntc by -5\repeat
		\the\mayacntc\mayacntc=\mayacnta
		\loop\ifnum\mayacntc>5\advance\mayacntc by -5 5\repeat}%
	\fi}%

\begin{document}

\maketitle

\section{Additive Zahlensysteme}

\begin{enumerate}[label=\Roman*)]
    \item
        \begin{equation*}
            400_{10} = 500_{10} - 100_{10} = D_R - C_R = \textit{CD}_R
        \end{equation*}
    \item
        \begin{equation*}
            133_{10} = 100_{10} + 3_{10} \cdot 10_{10} + 3_{10} \cdot 1_{10} = \textit{CXXXIII}_R
        \end{equation*}
        
\end{enumerate}

\section{Polyadische Zahlensysteme}
\begin{enumerate}[label=\alph*)]
    \item
        \begin{enumerate}[label=\Roman*)]
            \item $ 400_{10} = 1 \cdot 20^2 = \maya{400} $ 

            \item $ 133_{10} = 6 \cdot 20^1 + 13 = \maya{133} $ 

            \item $ 350_{10} = 17 \cdot 20^1 + 10 = \maya{350} $ 

            \item $ 622_{10} = 1 \cdot 20^2 + 11 \cdot 20^1 + 2 = \maya{622} $ 

            \item $ 6452_{10} = 16 \cdot 20^2 + 2 \cdot 20^1 + 12 = \maya{6452} $ 
        \end{enumerate}
\end{enumerate}

\section{Wir wollen dann bitte Zahlen!}
\begin{enumerate}[label=\alph*)]
	\item BADA55 \\
	Minimale Basis:  $b = 13$ \\
	Interpretationsmöglichkeiten: 4


	\item DEADC0DE \\
	Minimale Basis:  $b = 15$ \\
	Interpretationsmöglichkeiten: 2

	\item C0CAC01A \\
	Minimale Basis:  $b = 12$ \\
	Interpretationsmöglichkeiten: 5
\end{enumerate}

\section{Bitwertigkeit}
\begin{enumerate}[label=\alph*)]
    \item \emph{MSB} first
        \begin{enumerate}[label=\Roman*)]
            \item $ \$\sim*\#_H = 3201_4 = 3 \cdot 64 + 2 \cdot 16 + 1 = 222_{10} $

            \item $ \#\sim\sim\$_H = 1223_4 = 64 + 2 \cdot 16 + 2 \cdot 4 + 3 = 107_{10} $

            \item $ \#\$\$\sim*\sim_H = 133202_4 = \dots = 2018_{10} $
        \end{enumerate}

    \item \emph{LSB} first
        \begin{enumerate}[label=\Roman*)]
            \item $ \$\sim*\#_H = 1023_4 = 75_{10} $

            \item $ \#\sim\sim\$_H = 3221_4 = 233_{10} $

            \item $ \#\$\$\sim*\sim_H = 202331_4 = 2237_{10} $
        \end{enumerate}
\end{enumerate}

\end{document}
