\documentclass[a4paper]{article}

\usepackage[ngerman]{babel}
\usepackage[utf8]{inputenc}
\usepackage{enumitem}
\usepackage{amsmath}
\usepackage{array}
\newcolumntype{L}{>{$}c<{$}}


\title{Grundlagen der Rechnerarchitektur\\ Übungsblatt 5\\Gruppe 121\\}
\author{Jonas Otto\and Dominik Authaler}

\date{\today}

\begin{document}

\maketitle

\section*{Aufgabe 1}
Die Schaltalgebra ist eine spezielle Form der Boolschen Algebra. Während bei einer Boolschen Algebra nur festgelegt ist, dass sie auf einer bestimmten Trägermenge und mit bestimmten Operationen aufgebaut ist, sind diese in der Schaltalgebra genauer spezifiziert. Die Trägermenge der Schaltalgebra ist {0, 1} und die Operationen sind so gewählt, dass sie mittels Schaltungselementen realisiert werden können.

\section*{Aufgabe 2}
\begin{enumerate}[label=\alph*)]
    \item
        \begin{equation}
        \begin{aligned}
            g(x1, x2) &= \overline{\overline{x_1 \cdot x_2} \cdot x_1} \\
            &\stackrel{P8}{=} (x_1 \cdot x_2) + \overline{x_1} \\
            &\stackrel{P4'}{=} (x_1 + \overline{x_1}) \cdot (x_2 \cdot \overline{x_1}) \\
            &\stackrel{P9'}{=} 1 \cdot (x_2 \cdot \overline{x_1}) \\
            &\stackrel{P5}{=} x_2 \cdot \overline{x_1}
        \end{aligned}
        \end{equation}

    \item
        \begin{equation}
        \begin{aligned}
            h(x_1, x_2, x_3, x_4) &= (x_1 \cdot x_2) + (x_1 \cdot x_3) + x_1 \cdot (x_2 + x_3 \cdot x_4) + x_1 \\
            &\stackrel{P4}{=} x_1 \cdot (x_2 + x_3) + x_1 \cdot (x_2 + x_3 \cdot x_4) + x_1 \\
            &\stackrel{P4}{=} x_1 \cdot ((x_2 + x_3) + (x_2 + x_3 \cdot x_4) + 1) \\
            &\stackrel{P2', P6'}{=} x_1 \cdot ( x_2 + x_3 + x_2 + x_3 \cdot x_4) \\
            &\stackrel{P3'}{=} x_1 \cdot ( x_2 + x_3 + x_3 \cdot x_4) \\
            &\stackrel{P11'}{=} x_1 \cdot ( x_2 + x_3)
        \end{aligned}
        \end{equation}

    \item
        \begin{equation}
        \begin{aligned}
            k(x_1, x_2, x_3) &= ((x_1 + x_3 \cdot (x_2 + x_3)) \cdot 1) \cdot 1 \\
            &\stackrel{P5}{=} x_1 + x_3 \cdot (x_2 + x_3) \\
            &\stackrel{P11'}{=} x_1 + x_3
        \end{aligned}
        \end{equation}
\end{enumerate}

\section*{Aufgabe 3}
Die zu beweisenden Aussagen dieser Aufgabe sind auch als \textit{De Morgan'sche Regeln} bekannt.

\begin{enumerate}[label=\alph*)]
	\item
	\begin{equation}
		\begin{aligned}
		\overline{a + b} \stackrel{!}{=} \overline{a} \cdot \overline{b} \\
		\stackrel{P9, P9'}{\Longleftrightarrow} \overline{a} \cdot \overline{b} + \overline{\overline{a + b}} = 1 \wedge \overline{a} \cdot \overline{b} \cdot (\overline{\overline{a + b}}) = 0 
		\end{aligned}
	\end{equation}
	
	\begin{equation}
	\begin{aligned}
		&\stackrel{P9'}{\Longleftrightarrow} &\overline{a} \cdot \overline{b} + \overline{\overline{a+b}} = 1 \\
		&\stackrel{P7}{\Longleftrightarrow} &\overline{a} \cdot \overline{b} + (a + b) = 1 \\
		&\stackrel{P2'}{\Longleftrightarrow} &\overline{a} \cdot \overline{b} + a + b = 1 \\
		&\stackrel{Kleiner Konsens}{\Longleftrightarrow} &a + \overline{b} + b = 1 \\
		&\stackrel{P9'}{\Longleftrightarrow} &a + 1 = 1 \\
		&\stackrel{P6'}{\Longleftrightarrow} &1 = 1
	\end{aligned}
	\end{equation}
	
	\begin{equation}
	\begin{aligned}
		&\stackrel{P9'}{\Longleftrightarrow} &\overline{a} \cdot \overline{b} \cdot (\overline{\overline{a+b}}) = 0 \\
		&\stackrel{P7}{\Longleftrightarrow} &\overline{a} \cdot \overline{b} \cdot (a + b) = 1 \\
		&\stackrel{P4}{\Longleftrightarrow} &\overline{a} \cdot (\overline{b} \cdot a + \overline{b} \cdot b) = 0
		\\
		&\stackrel{P9}{\Longleftrightarrow} &\overline{a} \cdot (\overline{b} \cdot a + 0) = 0
		\\
		&\stackrel{P5'}{\Longleftrightarrow} &\overline{b} \cdot (\overline{a} \cdot a) = 0
		\\
		&\stackrel{P9}{\Longleftrightarrow} &\overline{b} \cdot 0 = 0
		\\
		&\stackrel{P6}{\Longleftrightarrow} &0 = 0
	\end{aligned}
	\end{equation}

	\clearpage
	\item 
	\begin{equation}
	\begin{aligned}
		\overline{a \cdot b} \stackrel{!}{=} \overline{a} + \overline{b} \\
		\stackrel{P9, P9'}{\Longleftrightarrow} \overline{a} + \overline{b} + \overline{\overline{a \cdot b}} = 1 \wedge \overline{a} + \overline{b} \cdot (\overline{\overline{a \cdot b}}) = 0 
		\end{aligned}
	\end{equation}
	
	\begin{equation}
	\begin{aligned}
		&\stackrel{P9}{\Longleftrightarrow} &(\overline{a} + \overline{b}) + \overline{\overline{a \cdot b}} = 1 \\
		&\stackrel{P7}{\Longleftrightarrow} &(\overline{a} + \overline{b}) + a \cdot b = 1 \\
		&\stackrel{P2'}{\Longleftrightarrow} &\overline{b} + \overline{a} + a \cdot b = 1 \\
		&\stackrel{Kleiner Konsens}{\Longleftrightarrow} &\overline{b} + \overline{a} + b = 1 \\
		&\stackrel{P9'}{\Longleftrightarrow} &1 + \overline{a} = 1 \\
		&\stackrel{P6'}{\Longleftrightarrow} &1 = 1
	\end{aligned}
	\end{equation}
	
	\begin{equation}
	\begin{aligned}
		&\stackrel{P9}{\Longleftrightarrow} &(\overline{a} + \overline{b}) \cdot \overline{\overline{a \cdot b}} = 0 \\
		&\stackrel{P7}{\Longleftrightarrow}	 &(\overline{a} + \overline{b}) \cdot a \cdot b = 0 \\
		&\stackrel{P4'}{\Longleftrightarrow} &(\overline{a} \cdot a + \overline{b} \cdot a) \cdot b = 0 \\
		&\stackrel{P9}{\Longleftrightarrow} &(0 + \overline{b} \cdot a) \cdot b = 0 \\
		&\stackrel{P5'}{\Longleftrightarrow} &(\overline{b} \cdot b) \cdot a = 0 \\
		&\stackrel{P9}{\Longleftrightarrow} &0 \cdot a = 0 \\
		&\stackrel{P6}{\Longleftrightarrow} &0 = 0
		\end{aligned}
	\end{equation}
\end{enumerate}

\section*{Aufgabe 4}
\begin{equation}
\begin{aligned}
    \overline{\overline{(x_1 \cdot \overline{(x_2 \cdot x_2)})} \cdot \overline{(\overline{(x_1 \cdot x_1)} \cdot x_2)}} \\
    \stackrel{P8}{=} (x_1 \cdot \overline{(x_2 \cdot x_2)}) + (\overline{(x_1 \cdot x_1)} \cdot x_2) \\
    \stackrel{P3}{=} (x_1 \cdot \overline{x_2}) + (\overline{x_1} \cdot x_2)
\end{aligned}
\end{equation}

\clearpage
\section*{Aufgabe 5}
\begin{enumerate}[label=\alph*)]
	\item 
	\begin{enumerate}[label=\roman*)]
		\item Wahrheitstabelle:
		\begin{figure}[h!]
			\centering
			\begin{tabular}{|c|c|c|c|}
				\hline
				$x_2$ & $x_1$ & $x_0$ & $f(x_2, x_1, x_0)$ \\
				\hline
				0 & 0 & 0 & 0 \\
				0 & 0 & 1 & 1 \\
				0 & 1 & 0 & 0 \\
				0 & 1 & 1 & 1 \\
				1 & 0 & 0 & 0 \\
				1 & 0 & 1 & 1 \\
				1 & 1 & 0 & 0 \\
				1 & 1 & 1 & 1 \\
				\hline
			\end{tabular}
		\caption{Wahrheitstabelle der Funktion $f(x_2, x_1, x_0)$}
		\end{figure}
	
		\item 
		Disjunktive kanonische Normalform:\\ 
		\begin{equation}
			f(x_2, x_1, x_0) = \overline{x_2x_1}x_0 + \overline{x_2}x_1x_0 + x_2\overline{x_1}x_0 + x_2x_1x_0
		\end{equation}
		
		Konjunktive kanonische Normalform: \\
		\begin{equation}
			f(x_2, x_1, x_0) = (\overline{x_2} + \overline{x_1} + \overline{x_0}) \cdot (\overline{x_2} + x_1 + \overline{x_0}) \cdot (x_2 + \overline{x_1} + \overline{x_0}) \cdot (x_2 + x_1 + \overline{x_0})
		\end{equation}
		
		\item 
		\begin{equation}
		\begin{aligned}
			f(x_2, x_1, x_0) &= & \overline{x_2x_1}x_0 + \overline{x_2}x_1x_0 + x_2\overline{x_1}x_0 + x_2x_1x_0 \\
			&\stackrel{P4}{=} & x_0 \cdot (\overline{x_2x_1} + \overline{x_2}x_1 + x_2\overline{x_1} + x_2x_1) \\
			&\stackrel{P1'}{=} & x_0 \cdot (\overline{x_2x_1} + x_2x_1 + \overline{x_2}x_1 + x_2\overline{x_1}) \\
			&\stackrel{P2'}{=} & x_0 \cdot ((\overline{x_2x_1} + x_2x_1) + ( \overline{x_2}x_1 + x_2\overline{x_1})) \\
			&\stackrel{P9'}{=} & x_0 \cdot (1 + 1) \\
			&\stackrel{P6'}{=} & x_0 \cdot 1 \\
			&\stackrel{P5}{=} & x_0 \\
		\end{aligned}
		\end{equation}
	\end{enumerate}

	\item 
	\begin{enumerate}[label=\roman*)]
		\item 
		\begin{equation}
		\begin{aligned}
			f(x_1, x_2, x_3) &= x_2 \\
			&= (\overline{x_1} + \overline{x_2} + \overline{x_3}) \cdot (\overline{x_1} + \overline{x_2} + x_3) \cdot (x_1 + \overline{x_2} + \overline{x_3}) \cdot (x_1 + \overline{x_2} + x_3)
		\end{aligned}
		\end{equation}
		
		\item 
		\begin{equation}
		\begin{aligned}
		f(x_1, x_2, x_3) &= x_1 \cdot \overline{x_2} + \overline{x_1} \cdot x_3 \\
		&= (\overline{x_1} + \overline{x_2} + \overline{x_3}) \cdot (\overline{x_1} + x_2 + \overline{x_3}) \cdot (x_1 + x_2 + \overline{x_3}) \cdot (x_1 + x_2 + x_3)
		\end{aligned}
		\end{equation}
	\end{enumerate}
\end{enumerate}
\clearpage
\section*{Aufgabe 6}
Die später verwendeten Beziehungen $\overline{a + a} = \overline{a}$ und $\overline{a \cdot a} = \overline{a}$ ergeben sich aus P3' bzw. P3.
\begin{enumerate}[label=\alph*)]
	\item $x_1 \oplus x_2 $
	\begin{enumerate}[label=(\roman*)]
		\item Mittels NOR:
		\begin{equation}
		\begin{aligned}
			&&x_1 \oplus x_2 \\
			&\stackrel{Definition}{=} &x_1 \cdot \overline{x_2} + \overline{x_1} \cdot x_2 \\
			&\stackrel{P7}{=} &\overline{\overline{x_1 \cdot \overline{x_2}}} + \overline{\overline{\overline{x_1} \cdot x_2}} \\
			&\stackrel{P8}{= } &\overline{\overline{x_1} + x_2} + \overline{x_1 + \overline{x_2}} \\
			&\stackrel{\overline{a + a} = \overline{a}}{=} & \overline{\overline{x_1 + x_1} + x_2} + \overline{x_1 + \overline{x_2 + x_2}} \\
			&\stackrel{\overline{\overline{a} + \overline{a}} = a}{=} &\overline{\overline{\overline{\overline{x_1 + x_1} + x_2} + \overline{x_1 + \overline{x_2 + x_2}}} + \overline{\overline{\overline{x_1 + x_1} + x_2} + \overline{x_1 + \overline{x_2 + x_2}}}}
		\end{aligned}
		\end{equation} 
		
		\item Mittels NAND:
		\begin{equation}
		\begin{aligned}
			&&x_1 \oplus x_2 \\
			&\stackrel{Definition}{=} &x_1 \cdot \overline{x_2} + \overline{x_1} \cdot x_2 \\
			&\stackrel{P7}{=} &\overline{\overline{x_1 \cdot \overline{x_2} + \overline{x_1} \cdot x_2}} \\
			&\stackrel{P8'}{=} &\overline{\overline{a \cdot \overline{b}} \cdot \overline{\overline{a} \cdot b}} \\
			&\stackrel{\overline{a} = \overline{a \cdot a}}{=} &\overline{\overline{a \cdot \overline{b \cdot b}} \cdot \overline{\overline{a \cdot a} \cdot b}}
		\end{aligned}
		\end{equation}
	\end{enumerate}

	\item $(x_1 \cdot \overline{x_2}) + (\overline{x_2} \cdot \overline{x_3}) + (x_3 \cdot \overline{x_0}) + (x_0 \cdot \overline{x_1})$
	\begin{equation}
	\begin{aligned}
		&&(x_1 \cdot \overline{x_2}) + (\overline{x_2} \cdot \overline{x_3}) + (x_3 \cdot \overline{x_0}) + (x_0 \cdot \overline{x_1}) \\
		&\stackrel{P7}{=} &\overline{\overline{(x_1 \cdot \overline{x_2}) + (\overline{x_2} \cdot \overline{x_3}) + (x_3 \cdot \overline{x_0}) + (x_0 \cdot \overline{x_1})}} \\
		&\stackrel{P2'}{=} &\overline{\overline{((x_1 \cdot \overline{x_2}) + (\overline{x_2} \cdot \overline{x_3})) + ((x_3 \cdot \overline{x_0}) + (x_0 \cdot \overline{x_1}))}} \\
		&\stackrel{P8'}{=} &\overline{\overline{((x_1 \cdot \overline{x_2}) + (\overline{x_2} \cdot \overline{x_3}))} \cdot \overline{((x_3 \cdot \overline{x_0}) + (x_0 \cdot \overline{x_1}))}} \\
		&\stackrel{P8'}{=} &\overline{\overline{(x_1 \cdot \overline{x_2})} \cdot \overline{(\overline{x_2} \cdot \overline{x_3})} \cdot \overline{(x_3 \cdot \overline{x_0})} \cdot \overline{(x_0 \cdot \overline{x_1})}} \\
		&\stackrel{\overline{a \cdot a} = \overline{a}}{=} &\overline{\overline{x_1 \cdot \overline{x_2 \cdot x_2}} \cdot \overline{\overline{x_2 \cdot x_2} \cdot \overline{x_3 \cdot x_3}} \cdot \overline{x_3 \cdot \overline{x_0 \cdot x_0}} \cdot \overline{x_0 \cdot \overline{x_1 \cdot x_1}}} \\
		&\stackrel{\overline{a \cdot 1} = \overline{a}}{=} &\overline{\overline{\overline{\overline{x_1 \cdot \overline{x_2 \cdot x_2}} \cdot \overline{\overline{x_2 \cdot x_2} \cdot \overline{x_3 \cdot x_3}}} \cdot 1} \cdot \overline{\overline{\overline{x_3 \cdot \overline{x_0 \cdot x_0}} \cdot \overline{x_0 \cdot \overline{x_1 \cdot x_1}}} \cdot 1}}
	\end{aligned}
	\end{equation}
	
\end{enumerate}
\end{document}
	
