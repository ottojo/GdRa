\documentclass[a4paper]{article}

\usepackage[ngerman]{babel}
\usepackage[utf8]{inputenc}
\usepackage{enumitem}
\usepackage{amsmath}
\usepackage{array}
\newcolumntype{L}{>{$}c<{$}}


\title{Grundlagen der Rechnerarchitektur\\ Übungsblatt 4\\Gruppe 121\\}
\author{Jonas Otto\and Dominik Authaler}

\date{\today}

\begin{document}

\maketitle

\section*{Aufgabe 1}
\begin{enumerate}[label=\alph*)]
\item vorzeichenbehaftet: $11000101010_2 = -1_{10} \cdot (1000101010)_2 = -554_{10}$ \\
	  b-Komplement: $11000101010_2 = -\text{NOT}(11000101001)_2 = -470_{10}$ \\
      b-1-Komplement: $11000101010_2 = -\text{NOT}(11000101010)_2 = -469_{10}$
\item vorzeichenbehaftet: $01111010_2 = 1_{10} \cdot (1111010)_2 = 122_{10}$ \\
      b-Komplement: $01111010_2 = 122_{10}$ \\
      b-1-Komplement: $01111010_2 = 122_{10}$
\item vorzeichenbehaftet: $1111111_2 = -1_{10} \cdot (111111)_2 = -63_{10}$ \\
      b-Komplement: $1111111_2 = -\text{NOT}(1111110)_2 = -1_{10}$ \\
      b-1-Komplement: $1111111_2 = -\text{NOT}(1111111)_2 = -0_{10}$
\end{enumerate}

\section*{Aufgabe 2}
\begin{enumerate}[label=\alph*)]
\item $10011010_2 \cdot 111001_2$\\
    \begin{equation}
    \begin{split}
        10011010 \cdot &111001\\
        \cline{1-2}
                1001101&0\\
                 100110&1\\
                  10011&010\\
                     10&011010\\
        \cline{1-2}
              010001001&001010\\
    \end{split}
    \end{equation}
\item $011101100001_2 \div 10110_2$
\begin{align*}
\begin{array} {ccccccccccccccccccc}
0&1&1&1&0&1&1&0&0&0&0&1&:&1&0&1&1&0& = 1010101_2 \hspace{1cm}\text{Rest: } 10011_2 \\
-&1&0&1&1&0&&&&&&&&&&&&& \\
\cline{1-8}
&0&0&1&1&1&1&0&&&&&&&&&&& \\
&&-&1&0&1&1&0&&&&&&&&&&& \\
\cline{4-10}
&&&0&1&0&0&0&0&0&&&&&&&&& \\
&&&-&&1&0&1&1&0&&&&&&&&& \\
\cline{5-12}
&&&&0&0&1&0&1&0&0&1&&&&&&& \\
&&&&&-&&1&0&1&1&0&&&&&&& \\
\cline{7-12}
&&&&&&0&1&0&0&1&1&&&&&&& \\
\end{array}
\end{align*}
\end{enumerate}

\section*{Aufgabe 3}
\begin{enumerate}[label=\alph*)]
\item $1,453125_{10} = 1,011101_2$ \\
	  \begin{align*}
	  	&0,453125 &\cdot& 2 &= &0,90625 &+ 0 \\
		&0,90625  &\cdot& 2 &= &0,8125 &+ 1 \\
		&0,8125   &\cdot& 2 &= &0,625 &+ 1 \\
		&0,625    &\cdot& 2 &= &0,25 &+ 1 \\
		&0,25     &\cdot& 2 &= &0,5 &+ 0 \\
		&0,5      &\cdot& 2 &= &0 &+ 1
      \end{align*}
		
\item $\frac{1}{3}_{10} = 0,010101\ldots$, exakte Darstellung als Festkommazahl nicht möglich,
    da $\frac{1}{3}$ nicht als endliche Summe von Zweierpotenzen darstellbar ist. 
 \begin{align*}
    &\frac{1}{3} &\cdot& 2 &= &\frac{2}{3} &+ 0 \\
    &\frac{2}{3} &\cdot& 2 &= &\frac{1}{3} &+ 1 \\
    &\frac{1}{3} &\cdot& 2 &= &\frac{2}{3} &+ 0 \\
    \end{align*}
    Aufgrund der Periodizität der Binärzahl sind auch 12 Bit für die Nachkommastellen nicht ausreichend für eine exakte Darstellung.
\end{enumerate}

\section*{Aufgabe 4}
\begin{enumerate}[label=\alph*)]
    \item $0100101010_2 \cdot 0000000010_2 = 0100101010_2 << 1 = 1001010100_2$
    \item $0000101_2 \cdot 000100_2 = 0000101_2 << 2 = 010100_2$
    \item $001110101001_2 \div 001000000000_2 = 001110101001_2 >> 9 =000000000001_2 $
    \item $01010111_2 \div 00001000_2 = 01010111.0_2 >> 3 = 00001010.11100000_2 = 10.875_{10}$
\end{enumerate}

\section*{Aufgabe 5}
\begin{tabular}{c|cccccccccc}
	Dezimal & 0 & 1 & 2 & 3 & 4 & 5 & 6 & 7 & 8 & 9 \\
	\hline
	BCD		& $0000_2$ & $0001_2$ & $0020_2$ & $0011_2$ & $0100_2$ & $0101_2$ & $0110_2$ & $0111_2$ & $1000_2$ & $1001_2$
\end{tabular}
\begin{enumerate}[label=\alph*)]
    \item $377_{10} = 001101110111_{BCD}$
    \item $17_{10} + 13_{10} = 00010111_{BCD} + 00010011_{BCD} = 0010 1010_{BCD} + 0110_{BCD} = 00110000_{BCD} = 30_{10}$
    \item $110_{10} + 99_{10} = 0001 0001 0000_{BCD} + 1001 1001_{BCD} = 0001 1010 1001_{BCD} + 0110 0000_{BCD} = 001000001001_{BCD} = 209_{10}$
    \item $3_{10} \cdot 4_{10} = 0011_{BCD} \cdot 0100_{BCD} = 00001100_2 + 0110_2 = 00010010_{BCD} = 12_{10}$
\end{enumerate}

\section*{Aufgabe 6}
\begin{enumerate}[label=\alph*)]
    \item $0101110_2 + 0110111_2 = 1100101_2 = -27$ Fehler: Resultat müsste als Summe zweier positiver Zahlen auch wieder positiv sein, allerdings überläuft das Register, sodass sich eine negative Zahl im Zweierkomplement ergibt.  
    \item $1011111_2 - 0110111_2 = 1011111_2 + 1001001_2 = 0101000_2$ Fehler: Resultat 8-bit (Resultat betragsmäßig zu groß)
\end{enumerate}

\section*{Aufgabe 7:}
Es gibt endliche dezimale Zahlen, deren binäre Repräsentation periodisch ist. Ein Beispiel hierfür ist $0,1_{10} = (0.0001100011...)_2:$
\begin{align*}
	0,1 \cdot 2 = 0,2 + 0 \\
	0,2 \cdot 2 = 0,4 + 0 \\
	0,4 \cdot 2 = 0,8 + 0 \\
	0,8 \cdot 2 = 0,6 + 1 \\
	0,6 \cdot 2 = 0,2 + 1
\end{align*}

\end{document}
	
