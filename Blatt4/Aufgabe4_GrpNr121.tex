\documentclass[a4paper]{article}

\usepackage[ngerman]{babel}
\usepackage[utf8]{inputenc}
\usepackage{enumitem}
\usepackage{amsmath}
\usepackage{array}
\newcolumntype{L}{>{$}c<{$}}


\title{Grundlagen der Rechnerarchitektur\\ Übungsblatt 4\\Gruppe 121\\}
\author{Jonas Otto\and Dominik Authaler}

\date{\today}

\begin{document}

\maketitle

\section*{Aufgabe 1}
\begin{enumerate}[label=\alph*)]
\item b-Komplement: $11000101010_2 = -\text{NOT}(11000101001)_2 = -470_{10}$ \\
      b-1-Komplement: $11000101010_2 = -\text{NOT}(11000101010)_2 = -469_{10}$
\item b-Komplement: $01111010_2 = 122_{10}$ \\
      b-1-Komplement: $01111010_2 = 122_{10}$
\item b-Komplement: $1111111_2 = -\text{NOT}(1111110)_2 = -1_{10}$ \\
      b-1-Komplement: $1111111_2 = -\text{NOT}(1111111)_2 = -0_{10}$
\end{enumerate}

\section*{Aufgabe 2: Multiplikation und Division}
\begin{enumerate}[label=\alph*)]
\item $10011010_2 \cdot 111001_2$\\
    \begin{equation}
    \begin{split}
        10011010 \cdot &111001\\
        \cline{1-2}
                1001101&0\\
                 100110&1\\
                  10011&010\\
                     10&011010\\
        \cline{1-2}
              010001001&001010\\
    \end{split}
    \end{equation}
\item $011101100001_2 \div 10110_2$
\end{enumerate}

\section*{Aufgabe 3}
\begin{enumerate}[label=\alph*)]
\item $1,453125_{10} = 1,011101_2$
\item $\frac{1}{3}_{10} = 0,010101\ldots$, exakte Darstellung als Festkommazahl nicht möglich,
    da $\frac{1}{3}$ nicht als endliche Summe von Zweierpotenzen darstellbar ist.
\end{enumerate}

\section*{Aufgabe 4}
\begin{enumerate}[label=\alph*)]
    \item $0100101010_2 \cdot 0000000010_2 = 0100101010_2 << 1 = 1001010100_2$
    \item $0000101_2 \cdot 000100_2 = 0000101_2 << 2 = 010100_2$
    \item $001110101001_2 \div 001000000000_2 = 001110101001_2 >> 9 =000000000001_2 $
    \item $01010111_2 \div 00001000_2 = 01010111_2 >> 3 = 00001010.11100000_2 = 10.875_{10}$
\end{enumerate}

\section*{Aufgabe 7: Knobelaufgabe}
Es gibt endliche dezimale Zahlen, deren binäre Repräsentation periodisch ist. Ein Beispiel hierfür ist $0,1_{10} = (0.0001100011...)_2:$
\begin{align*}
	0,1 \cdot 2 = 0,2 + 0 \\
	0,2 \cdot 2 = 0,4 + 0 \\
	0,4 \cdot 2 = 0,8 + 0 \\
	0,8 \cdot 2 = 0,6 + 1 \\
	0,6 \cdot 2 = 0,2 + 1
\end{align*}

\end{document}
	
