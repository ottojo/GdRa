\documentclass[a4paper]{article}

\usepackage[ngerman]{babel}
\usepackage[utf8]{inputenc}
\usepackage{enumitem}
\usepackage{amsmath}
\usepackage{array}
\newcolumntype{L}{>{$}c<{$}}


\title{Grundlagen der Rechnerarchitektur\\ Übungsblatt 4\\Gruppe 121\\}
\author{Jonas Otto\and Dominik Authaler}

\date{\today}

\begin{document}

\maketitle

\section{Aufgabe 1}
\begin{enumerate}[label=\alph*)]
\item b-Komplement: $11000101010_2 = -\text{NOT}(11000101001)_2 = -470_{10}$ \\
      b-1-Komplement: $11000101010_2 = -\text{NOT}(11000101010)_2 = -469_{10}$
\item b-Komplement: $01111010_2 = 122_{10}$ \\
      b-1-Komplement: $01111010_2 = 122_{10}$
\item b-Komplement: $1111111_2 = -\text{NOT}(1111110)_2 = -1_{10}$ \\
      b-1-Komplement: $1111111_2 = -\text{NOT}(1111111)_2 = -0_{10}$
\end{enumerate}

\section{Aufgabe 2: Multiplikation und Division}
\begin{enumerate}[label=\alph*)]
\item $10011010_2 \cdot 111001_2$\\
    \begin{equation}
    \begin{split}
        10011010 \cdot &111001\\
        \cline{1-2}
                1001101&0\\
                 100110&1\\
                  10011&010\\
                     10&011010\\
        \cline{1-2}
              010001001&001010\\
    \end{split}
    \end{equation}
\item $011101100001_2 \div 10110_2$
\end{enumerate}

\end{document}
	
