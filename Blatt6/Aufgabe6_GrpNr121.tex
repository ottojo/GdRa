\documentclass[a4paper]{article}

\usepackage[ngerman]{babel}
\usepackage[utf8]{inputenc}
\usepackage{enumitem}
\usepackage{amsmath}
\usepackage{array}
\newcolumntype{L}{>{$}c<{$}}


\title{Grundlagen der Rechnerarchitektur\\ Übungsblatt 6\\Gruppe 121\\}
\author{Jonas Otto\and Dominik Authaler}

\date{\today}

\begin{document}

\maketitle

\section*{Aufgabe 1}
\begin{enumerate}[label=\alph*)]
	\item Wahrheitstafel:
	\begin{figure}[h!]
		\centering
		\begin{tabular}{|c|c|c|}
			\hline
			Tag & Tag - 1 binär ($x_3x_2x_1x_0$) & Angeschaltete Äste \\
			\hline
			1 & 0000 & 1, 2, 3, 4, 5, 6, 7\\
			2 & 0001 & 2, 4, 6\\
			3 & 0010 & 2, 4, 6\\
			4 & 0011 & 1, 2, 4, 6\\
			5 & 0100 & 2, 4, 6\\
			6 & 0101 & 1, 2, 3, 4, 5, 6\\
			7 & 0110 & 1, 2, 3, 4, 5, 6\\
			8 & 0111 & 1, 2, 3, 4, 5, 6, 7\\
			9 & 1000 & 1, 3, 5\\
			10 & 1001 & 1, 2 3, 4, 5, 6\\
			11 & 1010 & 1, 3, 4, 5, 6\\
			12 & 1011 & 1, 3, 5\\
			13 & 1100 & 1, 3, 5\\
			14 & 1101 & 1, 2, 3, 4, 5, 6\\
			15 & 1110 & 1, 3, 5, 7\\
			16 & 1111 & 1, 2, 3, 5\\
			\hline
		\end{tabular}
		\caption{Wahrheitstabelle zur Ansteuerung der Segmente}
	\end{figure}
\end{enumerate}

\end{document}
	
